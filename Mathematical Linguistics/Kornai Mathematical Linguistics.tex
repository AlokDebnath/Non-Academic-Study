\documentclass{report}
\usepackage[utf8]{inputenc}
\usepackage{amsmath}
\usepackage{natbib}
\usepackage{url}

\title{Mathematical Linguistics - A. Kornai}
\author{Alok Debnath}
\date{September 2019}

\begin{document}

\maketitle
\begin{abstract}
  This is a summary of the textbook "Mathematical Linguistics" by A. Kornai.
  The book may be found on
  \url{http://www.helsinki.fi/esslli/courses/readers/K54.pdf}.
  Some parts of the book require some prerequisites which are non-crucial. I
  have attempted to capture those in the footnotes. For those which are crucial
  a relevant section has been added, which can be used. This is only a summary
  of the book, and I have skimmed over a lot of the parts, some of which may
  be of interest to the reader. If this is so, I recommend reading the 
  textbook.
\end{abstract}

\chapter{Introduction}

The aim of this text is to analyze the notions which are of linguistic
interest, such as phonology, morphology, syntax and semantics by using
"mathematical techniques". The author compares this to a study of mathematical
physics, where some branches or subfields are studied far more than others, not
because they are not interesting, more so beacuse there is no direct utility in
studying them over mathematical explanations of observable phenomenon.

\textbf{Definitions}: Definitions in mathematical linguistics follow three
basic steps, an \textit{ostensive} definition based on examples, an
\textit{extensive} definition which dealiantes the intended scope of the notion
and an \textit{intensive} definition, which exposes the underlying mechanism.

\textbf{Formalization}: Much like other branches of applied mathematics, the
problem of formalization of semi-formally or informally stated theories.
However, the choice of a formal definitions and structures is often hard. The
stochastic nature of language and linguistic rules is still a debate.

\textbf{Foundations}: Mathemaical linguistics has sets as the object of
interest, usually finite but sometimes denumberably infinite (see: set of
words). Due to emperical restrictions, however, denumerable generailzations
of finite objects such as $\omega$-words are rarely used. \footnote{$\omega$-
  words are words that can be written on an $\omega$-automaton, a finite 
automaton that runs on infinite strings as inputs. Buchi automata are examples
of the same.} For the purpose of this series, such constructions are not taken
into account. The study of language and mathematics have evolved rather
independently, the primary exception being Indian traditions of logic.

\textbf{Mesoscopy}: A mesoscopic system is one which is neither too small to be
appropriately studied by microscopic methods and tools, and yet is not large
enough for statistical approximations of macroscopic systems are necessarily
accurate. Natural language may be viewed as a mesoscopic systems, which is
supposedly governed by a thousands of rules, wheras the statistical methods of
macroscopic levels (Markov assumption, for example) is a reasonable
approximation of the behavior of Natural Language. Macroscopic techniques yeild
only generalizations of mesoscopic systems. Further, statistical quantities
of interest are linked "only very indirectly" to the object of study, which
requires very specific techniques in order to decide which parameters are of
interest and which are to be left unmodeled.

\end{document}
