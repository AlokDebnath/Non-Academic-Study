\documentclass{report}
\usepackage[utf8]{inputenc}
\usepackage{amsmath}
\usepackage{natbib}
\usepackage{url}

\title{Mathematical Linguistics - A. Kornai}
\author{Alok Debnath}
\date{September 2019}

\begin{document}

\maketitle
\begin{abstract}
  This is a summary of the textbook "Mathematical Linguistics" by A. Kornai.
  The book may be found on
  \url{http://www.helsinki.fi/esslli/courses/readers/K54.pdf}.
  Some parts of the book require some prerequisites which are non-crucial. I
  have attempted to capture those in the footnotes. For those which are crucial
  a relevant section has been added, which can be used. This is only a summary
  of the book, and I have skimmed over a lot of the parts, some of which may
  be of interest to the reader. If this is so, I recommend reading the
  textbook.
\end{abstract}

\chapter{Introduction}

The aim of this text is to analyze the notions which are of linguistic
interest, such as phonology, morphology, syntax and semantics by using
"mathematical techniques". The author compares this to a study of mathematical
physics, where some branches or subfields are studied far more than others, not
because they are not interesting, more so beacuse there is no direct utility in
studying them over mathematical explanations of observable phenomenon.

\textbf{Definitions}: Definitions in mathematical linguistics follow three
basic steps, an \textit{ostensive} definition based on examples, an
\textit{extensive} definition which dealiantes the intended scope of the notion
and an \textit{intensive} definition, which exposes the underlying mechanism.

\textbf{Formalization}: Much like other branches of applied mathematics, the
problem of formalization of semi-formally or informally stated theories.
However, the choice of a formal definitions and structures is often hard. The
stochastic nature of language and linguistic rules is still a debate.

\textbf{Foundations}: Mathemaical linguistics has sets as the object of
interest, usually finite but sometimes denumberably infinite (see: set of
words). Due to emperical restrictions, however, denumerable generailzations
of finite objects such as $\omega$-words are rarely used. \footnote{$\omega$-
  words are words that can be written on an $\omega$-automaton, a finite 
automaton that runs on infinite strings as inputs. Buchi automata are examples
of the same.} For the purpose of this series, such constructions are not taken
into account. The study of language and mathematics have evolved rather
independently, the primary exception being Indian traditions of logic.

\textbf{Mesoscopy}: A mesoscopic system is one which is neither too small to be
appropriately studied by microscopic methods and tools, and yet is not large
enough for statistical approximations of macroscopic systems are necessarily
accurate. Natural language may be viewed as a mesoscopic systems, which is
supposedly governed by a thousands of rules, wheras the statistical methods of
macroscopic levels (Markov assumption, for example) is a reasonable
approximation of the behavior of Natural Language. Macroscopic techniques yeild
only generalizations of mesoscopic systems. Further, statistical quantities
of interest are linked "only very indirectly" to the objects of interest and
therefore there are special techniques that must be employed in order to decide
which variables should be left unmodeled.

\chapter{The elements}

Fundamentally, enumaration of the set of objects useful to linguistics is an
one of the approaches to analyzing language. This enumeration, or generation
of the set of objects is an important topic of study. Along with generation,
another fundamental problem being solved is one of understanding set membership
and determining whether an object belongs to this set. This can be done by
generating the entire set and comparing the input against each one of the
generated objects, or by constructing a grammar which generates a language
against which the object in question can be compared.

\section{Generation}

Given a set of primitive elements $E$ and a set of non-procedural generation
rules $R$, it can be stated that $\forall x, y \in E,\, r \in R,\, \exists z
\text{ s.t. } z = r(x,y)$. Another perspective on the same is $z \rightarrow_r
xy$, \textit{i.e.} $z$ directly generates $x$ and $y$. The smallest collection
of objects closed under direct generation $r \in R$ which contains all the
elements of $E$ is called the set generated from $E$ by $R$.

Generative definitions have a notion of equality, which may be derivational
(stronger as it depends on how the object was generated, not an abstract
notion). Derivations can be mono- or multi-stratal, as in deriving via some
well-defined intermediate state. In order to understand generative defintions
better, a few examplesa are provided.

\subsection{Wang tilings}

\textbf{Problem Definition}: Let $C$ be a finite set of colors and $S$ be a
finite set of square tiles, colored on the edges according to a generative
function $e : S \rightarrow C^4 $. Arrange the tiles such that the adjacent
edges of the same tile are of the same color. Therefore:

$$
i, j \in \mathcal{Z} \text{, } u, v \in S \text{, and } \pi_n \in R \forall
n \in \{1, 2, 3, 4 \}\\

\therefore \text{ the four production rules are: } \\
\pi_3(e(i, j)) = \pi_1(e(i, j + 1)) \\
\pi_4(e(i, j)) = \pi_2(e(i + 1, j)) \\
\pi_2(e(i, j)) = \pi_4(e(i + 1, j)) \\
\pi_1(e(i, j)) = \pi_3(e(i, j + 1))
$$

\texxtbf{Discussion} The definition associated with Wang tiling is not entirely
generative, because of the following reasons:
\begin{enumerate}
  \item The definition relies on externally provided functions and objects,
  \item The rules are well-formedness conditions, not rules of production and
        in order to make that production rules, external constraints are
        necessary,
  \item The well formedness conditions are not recursively defined.
\end{enumerate}

\textbf{Extending the Problem}: It is recursively undecidable whther a given
inventory of tiles $E$ can yeild a Wang tiling.

\subsection{Groups as Generators}

\textbf{Problem Definition}: Let $E$ be a set of generators $g_1, g_2, ..., g_k
$ and their inverses $g_1^{-1}, g_2^{-1}, ..., g_k^{-1}$. Let $R$ contain the
product and cancellation rules. Formal products with usual rules of form a free
group over $k$, and without the inverses and cancellation rules, a free monoid
is formed over $k$ generators.

\textbf{Discussion}: It is in general undecidable whether a formal product of
generators is included in the kernel or not. Note that the defining relations
can be considered a part of production rules as well as equality relation.

\subsection{Herbrand Universes}

\textbf{Problem Definition}: First order languages consist of logical symbols
alongwith some constraints, and function and relation symbols. In Herbrand
universes, the elements $E$ are the onject constraints on the first order
logic. Rigourously,
$$
\forall x_i \in E, \\
f(\cdot) \text{ is a function constant of arity } n,\, f(x_i) \in E
$$

\textbf{Discussion}: Herbrand universes are purely formula based and is a
good example of generative definitions. However, usig first order logic
abstracts away many important properties of language.

\section{Axioms, Rules and Constraints}

One of the ways of contructing a system that undertands a language's
grammaticality is by propogating it through a set of rules or axioms, and
defining a group of combinatorial operations that preserve this notion of
grammaticality. Analogously, most transformations in grammars are seen as
applications of these rules followed by "cleaning up". Similarly the
characterization of the grammatical forms can be done using an axiomatic
system, the difference being that the starting constructions are abstract
such as words and a well defined hierarchy or representing the system (a
combination rule for the set of words, for example, is established).

\subsection{Balanced parenthesis}

\textbf{Problem Definition}: Given a series of parenthesis, there exists a well
formedness condition that determines that given a scoring function, the overall
score of the sequence of parenthesis never falls below a given value, and it is
a given value at the end of the series.

For example: If '(' is provided the number '$+1$' and ')' are provided '$-1$',
then the value of the sequence on addition should never drop below $0$ and the
value at the end of the sequence should be $0$.

\textbf{Discussion}: It is seen that the WFCs here rely on a transformation of
the problem into a more computable domain, where there exists a homomorphism
between both the elements and the rules that govern these elements (operations)
which are in the original set.

Instead of considering a score computation function in $\mathcal{Z}$, it is
better to consider a mapping to a finite structure $G$, with well understood
rules of combination, which allows disjuntive assignments. This makes the well
formedness constraints based entirely on the combiantion yeilding a desirable
result. The resulting desired combiantion allows for the constitution of a
certificate of membership to a grammar defined by a tranformation (in this case
$c: W \rightarrow 2^G$).

\subsection{Link Gramamrs}

Link grammars are a generalization of the Wang tiling problem. Link grammars
build on relations between pairs of words, which is similar to dependency
grammrs, but unlike dependency grammars, directionality is optional. The
introduction of link types and coloring multiplanar link grammars are
extensions of link grammars which makes them closely associted with categorial
grammars.

There is a fundamental difference between truth and grammaticality. The above
grammars defines the set of distinguished string and its cosets as well.
In categorial grammars, the object of inquiry extends from the strings of
words for which multiplication of the associated category yeilds the
distinguished generator but also those which might yeild another generator
or any specific word $G$. For example, it is possible to identify a grammatical
noun phrase which is not a grammatical sentence, while a gramamtical sentence
might have an ungramamtical noun phrase \footnote{This claim that the author,
makes in this sentence is dubious to say the very least, because an
ungrammtical noun phrase would not be a part of a grammtical sentene. The
example provided for this sentence is: "the house had seven gables" where the
noun phrase is not ungramamtical. However this is a minor point. The more
pressing matter at hand is the claim that grammaticality is not compositional,
which is absurd. However, it is to be noted that the sentence has no truth
value, provided grammaticality had some relation to the truth value.}

In logic, a model is only unique in the degernate case, as the Lowenheim-Skolem
theorems show that an inifinite model can produce as many isomorphic models as
the cardinals. Grammars on the other hand are unique constrained on
parameterization, in the sense that given an appropriate value of each
paramter used to determine each model, it defines a unique model.

\subsection{Undergeneration and Overgeneration}
The uniqueness of models results in overgeneration and undergeneration. If some
$c : W \rightarrow 2^G$ generates some string $w_1, w_2, w_3, ..., w_n \notin
M$ where $M$ is the model structure, and $w_1, w_2, w_3, ..., w_n$ belongs to
the yeild of $c$ then the grammar is undergenerating. Similarly, we define
overgeneration.

The generative power of a system can be based on the specificity of the
properties under study, and in other cases there is utility to overgeneration.
Constraint-based theories also have the same over- and under-generation issues.

\section{String Rewriting}
Defining:
\begin{enumerate}
  \item 
\end{enumerate}

\end{document}
